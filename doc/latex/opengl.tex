\section{OpenGL Interface}
\markright{\arabic{section}. OpenGL}
EusLisp has the interface to OpenGL, which is the standard
3D rendering package available on many platforms.
Though OpenGL was first implemented on SGI machines and was
dependent on their graphics hardware, 
the software emulation library, Mesa, has long been available.

To make use of the OpenGL interface,
euslisp should be invoked by ``eusgl'' instead of eus or eusx.
Machines on which eusgl is installed must have the opengl libraries,
namely libGL.so and libGLU.so. These may be symbolic links
either to SGI's openGL libraries or Mesa libraries, libMesaGL.so
and libMesaGLU.so.  Note that these libraries must be provided
in the shared object.


\subsection{OpenGL Extensions}

EusLisp's OpenGL interface has been implemented by subclassing
the viewsurface and viewer classes,
and by replacing the {\tt draw} function with {\tt gldraw}.
Classes for representing the color-material and lighting have also
been added.

\classdesc{glviewsurface}{geometry:viewsurface}{}{
glviewsurface defines a viewsurface class particularly for OpenGL.
Drawing requests to a viewer are interpreted to OpenGL requests.
}

\methoddesc{:create}{ \&key (x 0) (y 0) (size 256) (w size) (h size)
		      (title (string (gensym "GL")))
		      (displaymode (logior gl:AUX\_DOUBLE gl:AUX\_RGB gl:AUX\_DEPTH))
		      \&allow-other-keys}{
{\relax glviewsurface} is created and initialized.}

\methoddesc{:material}{ambient diffuse specular emission shininess}{
changes current material colors.}

\methoddesc{:color}{color-vector}{
changes the current color.}

\methoddesc{:3d-mode}{}{
射影変換を透視法射影モードにし、デプス検査、ライティングをイネーブルす
る。3次元モデル──シーンの描画に適したモードにする。
}
\methoddesc{:2d-mode}{}{
射影変換を正射影にし、デプス検査、ライティングをディスエーブルする。
2次元の描画要素の描画に適したモードにする。
}
\methoddesc{:clear}{mask \&key color}{
{\relax mask} で指定される、フレームバッファの内容を消去する。
{\relax color} は指定があれば、消去時に用いられる色を設定し直す。
}
\methoddesc{:init}{\&key \&allow-other-keys}{
初期化メソッド。
}
\methoddesc{:point}{x y}{
2次元モードで、点を描く。
}
\methoddesc{:line}{x1 y1 x2 y2}{
2次元モードで、直線を描く。
}
\methoddesc{:3d-point}{pos}{
3次元モードで点を描く。
}
\methoddesc{:3d-line}{start end}{
3次元モードで直線を描く。
}
\methoddesc{:draw-faces}{faces}{
3次元モードで面(ポリゴン)を描く。
}
\methoddesc{:rectangle}{x y w h}{
2次元モードで矩形を描く。
}
\methoddesc{:draw-point}{p}{
2次元モードで、点を描く。
}
\methoddesc{:draw-line}{from to}{
2次元モードで、直線を描く。
}
\methoddesc{:draw-rectangle}{}{
2次元モードで矩形を描く。
}
\classdesc{colormaterial}{object}{}{
モデルのマテリアル指定に用いるデータ構造。
}
\funcdesc{gl:draw-face}{face}{
ポリゴンの描画単位。凸でホールのない、ポリゴンは glVertex3fv\(\) を用
いて、そのまま描かれる。そうでないものは GLU ライブラリの tesselation 
機能を用いて描かれる。
}
\funcdesc{gl:draw-body}{abody}{
{\relax body} の描画単位。\\

{\relax body} の描画手順は一度、各面毎の描画要求の列として、その {\relax
body} を描く手順が OpenGL の Display List として定義され、実際の描画は
その後、その Display List を呼び出すことにより描画が行なわれる。\\

Display List の記録は {\relax body} の plist に attribute {\relax
:GL-DISPLAYLIST-ID} で put される。
}
\classdesc{linecolor}{object}{}{
点、直線の色指定に用いるデータ構造。
}

\funcdesc{cls}{}{
画面をクリアする。
}
\funcdesc{swapb}{}{
ダブルバッファ・モードのときのバッファ・スワップを行なう。
}
\funcdesc{resetperspective}{}{
現在の{\relax \*viewing\*} から視点、注視点位置を取り出し、GL のコンテキ
ストに反映させる。
}
\funcdesc{gldraw}{\&rest things}{
視点の再設定 {\relax (resetperspective)}、
フレームバッファのクリア {\relax (cls)}、
描画 {\relax (apply \#'draw things)}、
ダブルバッファの交換 {\relax (swapb)}
を順に行ない、シーンの描画を行なう。\\

things は左から右に順に処理され、その要素が、クラス colormaterial、
linecolor のインスタンスの時は、GL のカレントコンテキストの色が変更さ
れる。すなわち、(gldraw *color1* *obj1* *color2* *obj2*) では、
*obj1* が *color1* で描かれ、*obj2* が *color2* で描かれる。
}

\subsection{起動方法}
eusgl は OpenGL の補助ライブラリ aux、glx を用いるのため、環境変数 
DISPLAY、LD\_LIBRARY\_PATH が適当な値にセットされている必要がある。aux 
ライブラリは irix5 ではデフォルトでインストールされていないが、
/usr/share/src/OpenGL 下にソースコードがあるほか、
http://www.sgi.com/Technology/openGL/opengl.html または 
ftp://sgigate.sgi.com/pub/opengl/ 等から取得できる。\\

Mesa GL ライブラリを用いる場合は、LD\_LIBRARY\_PATH の変更のみ
で、切替ができる。詳細は /usr/local/eus/gl/src/README.Mesa を参照のこ
と。Mesa GL は、http://www.ssec.wisc.edu/\~brianp/Mesa.html または 
ftp://iris.ssec.wisc.edu/pub/Mesa/ 等から入手できる。\\

プログラム名 eusgl として起動することで、OpenGL 拡張機能つきの Euslisp
が立ち上がる。\\

eta3 マニピュレータを描くサンプル /usr/local/eus/gl/src/Sample.l が
用意してあり、

\begin{verbatim}
eusgl$ (load "/usr/local/eus/gl/src/Sample.l")
eusgl$ (load "view")
eusgl$ (gldraw *scene*)
\end{verbatim}

等とすると絵が表示できる。

